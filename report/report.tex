
\documentclass[journal]{IEEEtran}
\usepackage{blindtext}
\usepackage{graphicx}
\usepackage{hyperref}
\usepackage{polski}
\usepackage{float}
\usepackage[utf8x]{inputenc}
\usepackage{tikz} % To generate the plot from csv
\usepackage{pgfplots}
\usepackage{pgfplotstable}

\pgfplotsset{compat=newest} % Allows to place the legend below plot
\usepgfplotslibrary{units} % Allows to enter the units nicely


\begin{document}
\title{Przeszukiwanie zespołów muzycznych w poszukiwaniu ciekawych zaleźności}
\author{
    Mateusz Kruszyna \texttt{inf127252}\\
    \and
    Bartosz Górka \texttt{inf127228}\\
    \and
    Jarosław Skrzypczak \texttt{inf127265}
}
\markboth{Wyszukiwanie i przetwarzanie zasobów informacyjnych, 12 Czerwiec 2018}{}
\maketitle
\IEEEpeerreviewmaketitle

\section{Wprowadzenie}
Poniżej zostanie pokazane po krótce jak to zrobiliśmy i co nam wyszło ; )

\section{Pozyskanie danych}
Do pozyskania danych posłużył nam \textit{Apache solr} oraz \textit{Apache nutch}.
Z czternastu początkowych stron w domenie \textit{https://en.wikipedia.org/wiki/}
oraz skonfigurowaniu silnika szukającego tak, aby pomijał niepotrzebne linki
(zewnętrzne strony, linkki do edycji, linki do mediawiki itp...)
zostało pozykasne 1050 dokumnetów na temat zespołów muzycznych.
Niestety 42 linki z powyższej kolekcji prowadzą do podkategorii, które nie zostały
uwzględnione w przeszukiwaniu, lecz strony znajdujące się w tych podkategoriach już tak.

Drugim elementem było wgranie stworzonego indeksu do \textit{Apache Lucene}, w którym
dokonaliśmy analizy dokumentów. Z każdego dokumnetu pobraliśmy takie dane jak tytuł,
listę najczęstszych lokalizacji, listę najczęściej występującyh lat (tylko rok),
listę najczęściej występujących imion i nazwisk oraz
dopasowaliśmy najbardziej prawdobodobną kategorię do każdego dokumnetu
z prędzej wygenerowanej listy możlwiych kategorii.

Cały mechanizm pozyskiwania danych do wykresów zostanie krótko opisany poniżej.
\begin{itemize}
\item Tytuł został wyciągnięty z linku do strony na wikipedii (końcówka linku).
\item Zawartość dokumentu (plik html) został poddany tokenizacji, a następnie wykonane dalsze czynności
\item Z pomocą modułu do rozpoznawania lokalizacji \textit{OpenNLP}
zostały wybrane najczęściej występujące lokalizacje. Pominięta została lokalizacja "Random",
która naszym zdaniem była błednie rozpoznawana.
\item Kategorie zostały wybrane z przeszukiwanej bazy, dzięki charakterystycznym
schematom linków, które zawierały słowo kluczowe \textit{category:}.
Gdy w nazwie kategorii pojawiło się jedno ze słów zakazanych ("by","genre","navigational",
"(genre)","musicians","nationality","body"), to taka nazwa nie została uznana za właściwą nazwę.
Po takim zabiegu, aby wyszukać do jakiej kategorii należy dokument, dla każdego dokumentu
kategorie zostały rozbite na pojedyńcze słowa kluczowe i zliczone
sumaryczne wystąpienia każdego z tych słów w dokumencie.
\item Pozyskiwanie dat (rok) z tekstu niestety nie okazało się dobre przy wykorzystaniu gotowego modelu
z pakietu \textit{OpenNLP}. W zamian tego zostały wykorzystane proste wyrażenia regularne
do wykrycia dwóch zapisów zapisów lat (format XXXX oraz 'XX dla lat 19XX).
Ograniczonych od dołu przez rok 1000 a od góry przez 2051.
\item Wyszukiwanie imion i nazwisk zostało zaprogramowane dzięki  przygotowanemu modułowi z \textit{OpenNLP}.
Tym razem także wyniki nie były jednoznaczne i wkradało się dużo różnych nie właściwych form. Z tego powodu
wszelkie imiona i nazwiska, które zawierały jakieś słowo z listy ("Tools What","Permanent","Page",
"The","Music","In","Retrieved","American") nie zostały uwzględnione w zliczaniu.
\item Wszelkie słowa, które były filtrem w jakimkolwiek zliczaniu czy wybieraniu danych, zostały odkryte
z teksu poprzez pełne przeszukiwanie i zaobserwowanie danych anomalii.
\end{itemize}

Eksport danych do formatu csv został ograniczony do pierwszych 10 wystąpień danej zmiennej (czytelność wykresu),
a także do danych, które występują więcej niż 1 raz w tekście.

\section{Wykresy}
% if first time is "\newcommand" if others is "\renewcommand"

\newcommand{\namefile}{../Lucene_project/results/cat_loc/cat_loc_American_blues_rock.csv}
\newcommand{\titleplot}{Najcześciej występujące lokalizacje dla grupy 'American blues rock'}
\newcommand{\column}{Location}
\newcommand{\labx}{Lokacje}
\newcommand{\laby}{Wystąpienia}
\begin{figure}[h!]
  \begin{center}
      \pgfplotstableread[col sep=comma,]{\namefile}\datatable
  \begin{tikzpicture}
  \begin{axis}[
      ybar, ymin=0,
      ylabel=Occurence,
      xlabel=Location,
      xtick=data,
      xticklabels from table={\datatable}{Location},
      xticklabel style = {rotate=90,anchor=east}
      ]
      \addplot table [x expr=\coordindex, y=Occur,col sep=comma]{\datatable};
  \end{axis}
  \end{tikzpicture}
    \caption{\titleplot}
  \end{center}
\end{figure}

\renewcommand{\namefile}{../Lucene_project/results/cat_loc/cat_loc_American_blues.csv}
\renewcommand{\titleplot}{Najcześciej występujące lokalizacje dla grupy 'American blues'}
\begin{figure}[h!]
  \begin{center}
      \pgfplotstableread[col sep=comma,]{\namefile}\datatable
  \begin{tikzpicture}
  \begin{axis}[
      ybar, ymin=0,
      ylabel=Occurence,
      xlabel=Location,
      xtick=data,
      xticklabels from table={\datatable}{Location},
      xticklabel style = {rotate=90,anchor=east}
      ]
      \addplot table [x expr=\coordindex, y=Occur,col sep=comma]{\datatable};
  \end{axis}
  \end{tikzpicture}
    \caption{\titleplot}
  \end{center}
\end{figure}



\section{Conclusion}
\blindtext


\end{document}
