
\documentclass[journal]{IEEEtran}
\usepackage{blindtext}
\usepackage{graphicx}
\usepackage{hyperref}
\usepackage{polski}
\usepackage{float}
\usepackage[utf8x]{inputenc}
\usepackage{tikz} % To generate the plot from csv
\usepackage{pgfplots}
\usepackage{pgfplotstable}

\pgfplotsset{compat=newest} % Allows to place the legend below plot
\usepgfplotslibrary{units} % Allows to enter the units nicely


\begin{document}
\title{Przeszukiwanie zespołów muzycznych w poszukiwaniu ciekawych zaleźności}
\author{
    Mateusz Kruszyna \texttt{inf127252}\\
    \and
    Bartosz Górka \texttt{inf127228}\\
    \and
    Jarosław Skrzypczak \texttt{inf127265}
}
\markboth{Wyszukiwanie i przetwarzanie zasobów informacyjnych, 12 Czerwiec 2018}{}
\maketitle
\IEEEpeerreviewmaketitle

\section{Wprowadzenie}
Poniżej zostanie pokazane po krótce jak to zrobiliśmy i co nam wyszło ; )

\section{Pozyskanie danych}
Do pozyskania danych posłużył nam \textit{Apache solr} oraz \textit{Apache nutch}.
Z czternastu początkowych stron w domenie \textit{https://en.wikipedia.org/wiki/}
oraz skonfigurowanie silnika szukającego tak, aby pomijał niepotrzebne linki
(zewnętrzne strony, linkki do edycji, linki do mediawiki itp...)
zostało pozykasne 1050 dokumnetów na temat zespołów muzycznych.
Niestety 42 linki z powyższej kolekcji prowadzą do podkategorii, które to nie zostały
uwzględnione w przeszukiwaniu.

Drugim elementem było wgranie stworzonego indeksu do \textit{Apache Lucene}, w którym
dokonaliśmy analizy dokumentów. Z każdego dokumnetu pobraliśmy takie dane jak tytuł,
listę najczęstszych lokalizacji, listę najczęściej występującyh lat (tylko rok) oraz
dopasowaliśmy najbardziej prawdobodobną kategorię do każdego dokumnetu
z prędzej wygenerowanej listy możlwiych kategorii.

Krótko opisują cały mechanizm, to tytuł został wyciągnięty z linku do strony na wikipedii.
Zawartość dokumnetu (plik html) został poddany tokenizacji.
Z pomocą modułu do rozpoznawania lokalizacji zostały wybrane najczęściej występujące lokalizacje.
Kategorie zostały rozbite na pojedyńcze słowa kluczowe i zostały zliczone
sumaryczne wystąpienia każdego z tych słow w dokumencie.
Pozyskiwanie dat (rok) z tekstu niestety nie okazało się dobre przy wykorzystaniu gotowego modelu
z pakietu \textit{OpenNLP}. W zamian tego zostały wykorzystane wyrażenia regularne
do wykrycia prostych zapisów lat (format XXXX oraz 'XX dla lat 19XX).

Dla wszystkich powyższych danych zostały zastosowane proste metody eliminacji błędnych danych
np. liczba 0001 jako rok, liczba 9988 jako rok, wyrażenie 'Random' jako lokalizacja,
słowa 'by' lub 'genre' w nazwie kategorii.

Eksport danych do formatu csv został ograniczony do pierwszych 10 wystąpień danej zmiennej (czytelność wykresu),
a także do danych, które występują więcej niż 1 raz w tekście.

\section{Wykresy}
% if first time is "\newcommand" if others is "\renewcommand"

\newcommand{\namefile}{../Lucene_project/results/cat_loc/cat_loc_American_blues_rock.csv}
\newcommand{\titleplot}{Najcześciej występujące lokalizacje dla grupy 'American blues rock'}
\newcommand{\labx}{Lokacje}
\newcommand{\laby}{Wystąpienia}
\begin{figure}[h!]
  \begin{center}
      \pgfplotstableread[col sep=comma,]{\namefile}\datatable
  \begin{tikzpicture}
  \begin{axis}[
      ybar, ymin=0,
      ylabel=Occurence,
      xlabel=Location,
      xtick=data,
      xticklabels from table={\datatable}{Location},
      xticklabel style = {rotate=90,anchor=east}
      ]
      \addplot table [x expr=\coordindex, y=Occur,col sep=comma]{\datatable};
  \end{axis}
  \end{tikzpicture}
    \caption{\titleplot}
  \end{center}
\end{figure}

\renewcommand{\namefile}{../Lucene_project/results/cat_loc/cat_loc_American_blues.csv}
\renewcommand{\titleplot}{Najcześciej występujące lokalizacje dla grupy 'American blues'}
\begin{figure}[h!]
  \begin{center}
      \pgfplotstableread[col sep=comma,]{\namefile}\datatable
  \begin{tikzpicture}
  \begin{axis}[
      ybar, ymin=0,
      ylabel=Occurence,
      xlabel=Location,
      xtick=data,
      xticklabels from table={\datatable}{Location},
      xticklabel style = {rotate=90,anchor=east}
      ]
      \addplot table [x expr=\coordindex, y=Occur,col sep=comma]{\datatable};
  \end{axis}
  \end{tikzpicture}
    \caption{\titleplot}
  \end{center}
\end{figure}



\section{Conclusion}
\blindtext


\end{document}
