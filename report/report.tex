
\documentclass[journal]{article}
\usepackage{blindtext}
\usepackage{graphicx}
\usepackage{hyperref}
\usepackage{polski}
\usepackage{float}
\usepackage[utf8x]{inputenc}
\usepackage{tikz} % To generate the plot from csv
\usepackage{pgfplots}
\usepackage{pgfplotstable}
\usepackage{geometry}
 \geometry{
 a4paper,
 total={170mm,257mm},
 left=25mm,
 top=25mm,
 right=25mm,
 bottom=25mm,
 }
\pgfplotsset{compat=newest} % Allows to place the legend below plot
\usepgfplotslibrary{units} % Allows to enter the units nicely


\begin{document}
\title{Przeszukiwanie zespołów muzycznych w poszukiwaniu ciekawych zależności}
\author{
    Mateusz Kruszyna \texttt{inf127252}\\
    \and
    Bartosz Górka \texttt{inf127228}\\
    \and
    Jarosław Skrzypczak \texttt{inf127265}
}
\markboth{Wyszukiwanie i przetwarzanie zasobów informacyjnych, 12 Czerwiec 2018}{}
\maketitle
%\IEEEpeerreviewmaketitle

\section{Wprowadzenie}
W realizowanym eksperymencie podjęliśmy się analizy zależności, jakie można odkryć
 w danych dotyczących zespołów muzycznych.
Postanowiliśmy ograniczyć się do dwóch kategorii,
aby przedstawić wybrane charakterystyki, bez nadmiarowych analiz.

\section{Pozyskanie danych}

Do pozyskania danych do analizy zastosowano  \textit{Apache Solr} oraz  \textit{Apache Nutch}.
Postanowiono wykorzystać strony  \textit{Wikipedii} i czternaście z nich wstawić
do zbioru startowych odnośników. Koniecznym było zagwarantowanie poprawnych
ustawień w silnikach przetwarzających, aby ignorowały one niepotrzebne odnośniki
takie jak zewnętrzne strony, załączniki, historie edycji bądź też odnośniki do mediawiki.

Po przygotowaniu środowiska, udało się pozyskać informacje w postaci 1050 dokumentów
na temat zespołów muzycznych. Niestety, po wstępnej analizie zebranych informacji
okazało się, że część odnośników prowadziła do podkategorii, których zespoły zostały
uwzględnione w bazie. Postanowiono wyeliminować strony podkategorii.

Tak uzyskany indeks (zbudowanego z wykorzystaniem  \textit{Apache Solr} +  \textit{Apache Nutch})
poddano analizie z wykorzystaniem  \textit{Apache Lucene}. Z każdego dokumentu pobrano następujące informacje:
\begin{itemize}
\item Tytuł, wycięty z adresu URL strony
\item Zawartość dokumentu (plik html) został poddany tokenizacji
\item Lokalizacje - uzyskane za pomocą modułu do rozpoznawania lokalizacji \textit{OpenNLP}.
\item Kategorie zostały wybrane z przeszukiwanej bazy, dzięki charakterystycznym schematom linków,
które zawierały słowo kluczowe `category:`
\item Pozyskiwanie dat (rok) z tekstu niestety nie okazało się poprawne przy wykorzystaniu
gotowego modelu z pakietu \textit{OpenNLP}. Jako alternatywę wykorzystano wyrażenia regularne
do wykrycia dwóch zapisów zapisów lat (format XXXX oraz 'XX dla lat 19XX).
Ponadto ograniczono lata do zakresu [1000, 2051]
\item Wyszukiwanie imion i nazwisk zostało zrealizowane dzięki wykorzystaniu \textit{OpenNLP}.
 Także tym razem wyniki nie były jednoznaczne i zawierały wiele niewłaściwych form.
Z tego powodu wszelkie imiona i nazwiska, które zawierały jakieś słowo z listy
("Tools What", "Permanent", "Page", "The", "Music", "In", "Retrieved", "American") nie zostały uwzględnione w zliczaniu.
\item Wszelkie słowa, które były filtrem w jakimkolwiek zliczaniu
czy wybieraniu danych, zostały odkryte z teksu poprzez pełne przeszukiwanie
i zaobserwowanie danych anomalii.
\end{itemize}

\section{Wstępne oczyszczenie danych}
W lokalizacjach pominięto `Random` którą uznano jako błędnie rozpoznawaną.
Gdy w nazwie kategorii pojawiło się jedno ze słów zakazanych
("by", "genre", "navigational", "(genre)", "musicians", "nationality", "body"),
 to taka nazwa nie była brana pod uwagę. Po takim zabiegu, aby wyszukać do jakiej
kategorii należy dokument, dla każdego dokumentu słowa kluczowe w postaci
pojedynczych wyrazów z kategorii zostały zliczone sumaryczne wystąpienia
każdego z tych słów w dokumencie.

\section{Analiza wykresów}
% if first time is "\newcommand" if others is "\renewcommand"

\newcommand{\namefile}{../Lucene_project/results/cat_loc/cat_loc_American_blues_rock.csv}
\newcommand{\titleplot}{Najcześciej występujące lokalizacje dla grupy 'American blues rock'}
\newcommand{\column}{Location}
\newcommand{\labx}{Lokacje}
\newcommand{\laby}{Wystąpienia}
\begin{figure}[h!]
  \begin{center}
      \pgfplotstableread[col sep=comma,]{\namefile}\datatable
  \begin{tikzpicture}
  \begin{axis}[
      ybar, ymin=0,
      ylabel=Occurence,
      xlabel=Location,
      xtick=data,
      xticklabels from table={\datatable}{Location},
      xticklabel style = {rotate=90,anchor=east}
      ]
      \addplot table [x expr=\coordindex, y=Occur,col sep=comma]{\datatable};
  \end{axis}
  \end{tikzpicture}
    \caption{\titleplot}
  \end{center}
\end{figure}

Zgodnie z oczekiwaniami, dla zespołu odnoszącego się do Stanów Zjednoczonych,
taka fraza (United States) pojawiała się najczęściej. Co jest zaskakującego -
bardzo wiele wystąpień odnotowała Europa. Również i ostatnia w prezentowanym
zestawieniu Wielka Brytania (England) cieszyła się popularnością mimo braku powiązania wprost.
Być może fraza `American blues rock` odniosła się także do koncertów w Europie
 i ich powiązań z europejskimi wykonawcami. New York (pozycja 3), California (pozycja 4),
 Los Angeles (pozycja 6), New York City (pozycja 7) to przykłady lokalizacji których
 spodziewaliśmy się osiągnąć w analizie lokalizacji.

 \renewcommand{\namefile}{../Lucene_project/results/cat_loc/cat_loc_American_blues.csv}
 \renewcommand{\titleplot}{Najcześciej występujące lokalizacje dla grupy 'American blues'}
\begin{figure}[h!]
  \begin{center}
      \pgfplotstableread[col sep=comma,]{\namefile}\datatable
  \begin{tikzpicture}
  \begin{axis}[
      ybar, ymin=0,
      ylabel=Occurence,
      xlabel=Location,
      xtick=data,
      xticklabels from table={\datatable}{Location},
      xticklabel style = {rotate=90,anchor=east}
      ]
      \addplot table [x expr=\coordindex, y=Occur,col sep=comma]{\datatable};
  \end{axis}
  \end{tikzpicture}
    \caption{\titleplot}
  \end{center}
\end{figure}

Podobnie jak w przypadku analizy dla American blues rock, również i American blues
 w znaczącej większości odnoszą się do Stanów Zjednoczonych. W tym przypadku mamy
zmianę ustawienia lokalizacji w stosunku do wcześniej charakteryzowanego Rysunek 1.
Co ciekawe, w zestawieniu pojawiła się fraza `Rock` która jest dość sporna - teoretycznie
może odnosić się zarówno do lokalizacji (góra / skała jak i stylu jakim jest rock).

\renewcommand{\namefile}{../Lucene_project/results/cat_loc/cat_loc_American_instrumental.csv}
\renewcommand{\titleplot}{Najcześciej występujące lokalizacje dla grupy 'American instrumental'}
\begin{figure}[h!]
  \begin{center}
      \pgfplotstableread[col sep=comma,]{\namefile}\datatable
  \begin{tikzpicture}
  \begin{axis}[
      ybar, ymin=0,
      ylabel=Occurence,
      xlabel=Location,
      xtick=data,
      xticklabels from table={\datatable}{Location},
      xticklabel style = {rotate=90,anchor=east}
      ]
      \addplot table [x expr=\coordindex, y=Occur,col sep=comma]{\datatable};
  \end{axis}
  \end{tikzpicture}
    \caption{\titleplot}
  \end{center}
\end{figure}

W przypadku analizy lokalizacji jakie zostały wyszukane dla `American instrumental`
poza oczywistymi trendami zaprezentowanymi również w przypadku Rysunek 1 oraz Rysunek 2,
 mamy niespodziewane wystąpienie `Japan` czyli odniesienia do Japonii.
 Ciężko scharakteryzować przyczynę tego wystąpienia, może to być nawiązanie do
koncertów w tamtym kraju, bądź także i wszelkie wzorce czerpane z tego kraju.

\renewcommand{\namefile}{../Lucene_project/results/cat_loc/cat_loc_British_instrumental.csv}
\renewcommand{\titleplot}{Najcześciej występujące lokalizacje dla grupy 'British instrumental'}
\begin{figure}[h!]
  \begin{center}
      \pgfplotstableread[col sep=comma,]{\namefile}\datatable
  \begin{tikzpicture}
  \begin{axis}[
      ybar, ymin=0,
      ylabel=Occurence,
      xlabel=Location,
      xtick=data,
      xticklabels from table={\datatable}{Location},
      xticklabel style = {rotate=90,anchor=east}
      ]
      \addplot table [x expr=\coordindex, y=Occur,col sep=comma]{\datatable};
  \end{axis}
  \end{tikzpicture}
    \caption{\titleplot}
  \end{center}
\end{figure}

Ciekawe wzorce odnoszące się do lokalizacji pojawiają się dla `British instrumental`.
Tutaj przeważa wystąpienie Londynu oraz odnośniki do europejskich nazw tj. Wielka Brytania,
Europa, Zjednoczone Królestwo, Francja czy Irlandia.
Na uwagę zasługują jednak odnośniki do Kalifornii, Stanów Zjednoczonych oraz Nowego Jorku -
które mogą wynikać z koncertowania na terenie USA przez zespoły.
Jednakże najbardziej zaskakującym jest ponownie odnośnik do Japonii.
Podobnie jak w przypadku analizy dla Rysunek 3. Tutaj również ciężko scharakteryzować
 dlaczego tak licznie wystąpił odnośnik do kraju kwitnącej wiśni.

\renewcommand{\namefile}{../Lucene_project/results/cat_loc/cat_loc_Canadian_Celtic.csv}
\renewcommand{\titleplot}{Najcześciej występujące lokalizacje dla grupy 'Canadian Celtic'}
\begin{figure}[h!]
  \begin{center}
      \pgfplotstableread[col sep=comma,]{\namefile}\datatable
  \begin{tikzpicture}
  \begin{axis}[
      ybar, ymin=0,
      ylabel=Occurence,
      xlabel=Location,
      xtick=data,
      xticklabels from table={\datatable}{Location},
      xticklabel style = {rotate=90,anchor=east}
      ]
      \addplot table [x expr=\coordindex, y=Occur,col sep=comma]{\datatable};
  \end{axis}
  \end{tikzpicture}
    \caption{\titleplot}
  \end{center}
\end{figure}

Analizując `Canadian Celtic` możemy zauważyć jak wielkie znaczenie ma drugie określenie we
frazie czyli nawiązanie do Celtów. W tym przypadku zgodnie z oczekiwaniami,
na dwóch pierwszych miejscach mamy Kanadę oraz Stany Zjednoczone.
Jednakże kolejne dwie pozycji (oraz dodatkowo Dublin) to ewidentny
przykład nawiązania do krajów związanych z kulturą celtycką czyli Irlandii oraz Szkocji.

\renewcommand{\namefile}{../Lucene_project/results/cat_loc/cat_loc_Celtic.csv}
\renewcommand{\titleplot}{Najcześciej występujące lokalizacje dla grupy 'Celtic'}
\begin{figure}[h!]
  \begin{center}
      \pgfplotstableread[col sep=comma,]{\namefile}\datatable
  \begin{tikzpicture}
  \begin{axis}[
      ybar, ymin=0,
      ylabel=Occurence,
      xlabel=Location,
      xtick=data,
      xticklabels from table={\datatable}{Location},
      xticklabel style = {rotate=90,anchor=east}
      ]
      \addplot table [x expr=\coordindex, y=Occur,col sep=comma]{\datatable};
  \end{axis}
  \end{tikzpicture}
    \caption{\titleplot}
  \end{center}
\end{figure}

Dla samej frazy `Celtic` mamy bardzo liczne nawiązania do Irlandii oraz Szkocji.
Ponadto pojawiają się europejskie lokalizacje. Dużym zaskoczeniem niewątpliwie
jest Australia, która raczej nie kojarzy się w ogóle z omawianym określeniem.

\renewcommand{\namefile}{../Lucene_project/results/cat_loc/cat_loc_Doo-wop.csv}
\renewcommand{\titleplot}{Najcześciej występujące lokalizacje dla grupy 'Doo-wop'}
\begin{figure}[h!]
  \begin{center}
      \pgfplotstableread[col sep=comma,]{\namefile}\datatable
  \begin{tikzpicture}
  \begin{axis}[
      ybar, ymin=0,
      ylabel=Occurence,
      xlabel=Location,
      xtick=data,
      xticklabels from table={\datatable}{Location},
      xticklabel style = {rotate=90,anchor=east}
      ]
      \addplot table [x expr=\coordindex, y=Occur,col sep=comma]{\datatable};
  \end{axis}
  \end{tikzpicture}
    \caption{\titleplot}
  \end{center}
\end{figure}

Dla grupy `Doo-wop` obserwujemy głównie nawiązania do Nowego Jorku oraz Stanów Zjednoczonych.
Ponadto pojawia się Bronx czyli być może jest to grupa z tego obszaru.
 Zaskakującymi są jednakże bardzo liczne wystąpienia Love, Love „ oraz Rock - które lokalizacjami nie są.
Wykorzystany model do rozpoznawania lokalizacji (NLP) zwrócił takie wyniki, choć ich jakość wydaje się wątpliwa.

\renewcommand{\namefile}{../Lucene_project/results/cat_loc/cat_loc_Electronic.csv}
\renewcommand{\titleplot}{Najcześciej występujące lokalizacje dla grupy 'Electronic'}
\begin{figure}[h!]
  \begin{center}
      \pgfplotstableread[col sep=comma,]{\namefile}\datatable
  \begin{tikzpicture}
  \begin{axis}[
      ybar, ymin=0,
      ylabel=Occurence,
      xlabel=Location,
      xtick=data,
      xticklabels from table={\datatable}{Location},
      xticklabel style = {rotate=90,anchor=east}
      ]
      \addplot table [x expr=\coordindex, y=Occur,col sep=comma]{\datatable};
  \end{axis}
  \end{tikzpicture}
    \caption{\titleplot}
  \end{center}
\end{figure}

Dla `Electronic` obserwujemy nawiązania do krajów, a w dwóch przypadkach do miast
(Londyn oraz Los Angeles). Trend na tym wykresie jest najmniej zaskakujący ze
wszystkich do tej pory scharakteryzowanych.

\renewcommand{\column}{Year}
\renewcommand{\labx}{Czas (rok)}
\renewcommand{\laby}{Wystąpienia}
\renewcommand{\namefile}{../Lucene_project/results/cat_time/cat_time_Doo-wop.csv}
\renewcommand{\titleplot}{Najcześciej występujące lata dla grupy 'Doo-wop'}
\begin{figure}[h!]
  \begin{center}
      \pgfplotstableread[col sep=comma,]{\namefile}\datatable
  \begin{tikzpicture}
  \begin{axis}[
      ybar, ymin=0,
      ylabel=Occurence,
      xlabel=Location,
      xtick=data,
      xticklabels from table={\datatable}{Location},
      xticklabel style = {rotate=90,anchor=east}
      ]
      \addplot table [x expr=\coordindex, y=Occur,col sep=comma]{\datatable};
  \end{axis}
  \end{tikzpicture}
    \caption{\titleplot}
  \end{center}
\end{figure}

Analizując lata dla poszczególnych grup można zaobserwować bardzo zbliżone do siebie wystąpienia,
ewentualnie minimalna przewaga któregoś z lat. W przypadku tej grupy mamy jednakże
bardzo liczne wystąpienia lat .50 i .60 XX wieku - najprawdopodobniej
 szczyt popularności przypadał właśnie na ten okres.

 \renewcommand{\namefile}{../Lucene_project/results/cat_name/cat_name_American_blues_rock.csv}
 \renewcommand{\titleplot}{Najcześciej występujące imiona i nazwiska dla grupy 'American blues rock'}
 \renewcommand{\column}{Name}
 \renewcommand{\labx}{Imiona i nazwiska}
 \renewcommand{\laby}{Wystąpienia}
 \begin{figure}[h!]
  \begin{center}
      \pgfplotstableread[col sep=comma,]{\namefile}\datatable
  \begin{tikzpicture}
  \begin{axis}[
      ybar, ymin=0,
      ylabel=Occurence,
      xlabel=Location,
      xtick=data,
      xticklabels from table={\datatable}{Location},
      xticklabel style = {rotate=90,anchor=east}
      ]
      \addplot table [x expr=\coordindex, y=Occur,col sep=comma]{\datatable};
  \end{axis}
  \end{tikzpicture}
    \caption{\titleplot}
  \end{center}
\end{figure}

 \renewcommand{\namefile}{../Lucene_project/results/cat_name/cat_name_American_blues.csv}
 \renewcommand{\titleplot}{Najcześciej występujące imiona i nazwiska dla grupy 'American blues'}
 \begin{figure}[h!]
  \begin{center}
      \pgfplotstableread[col sep=comma,]{\namefile}\datatable
  \begin{tikzpicture}
  \begin{axis}[
      ybar, ymin=0,
      ylabel=Occurence,
      xlabel=Location,
      xtick=data,
      xticklabels from table={\datatable}{Location},
      xticklabel style = {rotate=90,anchor=east}
      ]
      \addplot table [x expr=\coordindex, y=Occur,col sep=comma]{\datatable};
  \end{axis}
  \end{tikzpicture}
    \caption{\titleplot}
  \end{center}
\end{figure}

 \renewcommand{\namefile}{../Lucene_project/results/cat_name/cat_name_American_instrumental.csv}
 \renewcommand{\titleplot}{Najcześciej występujące imiona i nazwiska dla grupy 'American instrumental'}
 \begin{figure}[h!]
  \begin{center}
      \pgfplotstableread[col sep=comma,]{\namefile}\datatable
  \begin{tikzpicture}
  \begin{axis}[
      ybar, ymin=0,
      ylabel=Occurence,
      xlabel=Location,
      xtick=data,
      xticklabels from table={\datatable}{Location},
      xticklabel style = {rotate=90,anchor=east}
      ]
      \addplot table [x expr=\coordindex, y=Occur,col sep=comma]{\datatable};
  \end{axis}
  \end{tikzpicture}
    \caption{\titleplot}
  \end{center}
\end{figure}

Analizując jednocześnie odniesienia do tych trzech amerykańskich grup możemy zaobserwować ten sam wzorzec -
 jakie pierwsze trzy najliczniejsze imiona pojawiają się męskie `John`, `David` oraz kobiece `Tina`.
Starając się zorientować nad poprawnością wyników, warto skupić się i zastanowić samemu -
jakich artystów się kojarzy. Zdecydowanie te imiona będą najpopularniejsze (najprawdopodobniej).

\section{Podsumowanie}
Zgodnie z oczekiwaniami, w danych dotyczących analizowanych zespołów muzycznych
 występuje wiele ciekawych i zaskakujących trendów. Dzięki możliwości przeprowadzenia
analizy z wykorzystaniem nowoczesnych narzędzi jakim jest \textit{Apache Lucene},
\textit{Apache Solr} czy \textit{OpenNLP} można było zbadać te zależności.
Był to również doskonały trening programistyczny w celu rozwiązania problemu analizy dużej grupy danych.
\end{document}
