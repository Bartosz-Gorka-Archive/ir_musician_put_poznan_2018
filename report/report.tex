
\documentclass[journal]{IEEEtran}
\usepackage{blindtext}
\usepackage{graphicx}
\usepackage{hyperref}
\usepackage{polski}
\usepackage[utf8x]{inputenc}
\usepackage{tikz} % To generate the plot from csv
\usepackage{pgfplots}
\usepackage{pgfplotstable}

\pgfplotsset{compat=newest} % Allows to place the legend below plot
\usepgfplotslibrary{units} % Allows to enter the units nicely


\begin{document}
\title{Przeszukiwanie zespołów muzycznych w poszukiwaniu ciekawych zaleźności}
\author{
    Mateusz Kruszyna \texttt{inf127252}\\
    \and
    Bartosz Górka \texttt{inf127228}\\
    \and
    Jarosław Skrzypczak \texttt{inf127265}
}
\markboth{Wyszukiwanie i przetwarzanie zasobów informacyjnych, 12 Czerwiec 2018}{}
\maketitle
\IEEEpeerreviewmaketitle

\section{Wprowadzenie}
Poniżej zostanie pokazane po krótce jak to zrobiliśmy i co nam wyszło ; )

\section{Pozyskanie danych}
Do pozyskania danych posłużył nam \textit{solr} oraz \textit{nutch}.
Z parunastu początkowych stron na \textit{wikipedii} pomijając niepotrzebne linki
 zostało pozykasne 1050 dokumnetów na temat zespołów muzycznych.

\section{Wykresy}
% if first time is "\newcommand" if others is "\renewcommand"

\newcommand{\namefile}{../Lucene_project/results/cat_loc/cat_loc_American_blues_rock.csv}
\newcommand{\titleplot}{My first plot}
\begin{figure}[h!]
  \begin{center}
      \pgfplotstableread[col sep=comma,]{\namefile}\datatable
  \begin{tikzpicture}
  \begin{axis}[
      ybar, ymin=0,
      ylabel=Occurence,
      xlabel=Location,
      xtick=data,
      xticklabels from table={\datatable}{Location},
      xticklabel style = {rotate=90,anchor=east}
      ]
      \addplot table [x expr=\coordindex, y=Occur,col sep=comma]{\datatable};
  \end{axis}
  \end{tikzpicture}
    \caption{\titleplot}
  \end{center}
\end{figure}


\renewcommand{\namefile}{../Lucene_project/results/cat_loc/cat_loc_American_blues.csv}
\renewcommand{\titleplot}{My first plot22}
\begin{figure}[h!]
  \begin{center}
      \pgfplotstableread[col sep=comma,]{\namefile}\datatable
  \begin{tikzpicture}
  \begin{axis}[
      ybar, ymin=0,
      ylabel=Occurence,
      xlabel=Location,
      xtick=data,
      xticklabels from table={\datatable}{Location},
      xticklabel style = {rotate=90,anchor=east}
      ]
      \addplot table [x expr=\coordindex, y=Occur,col sep=comma]{\datatable};
  \end{axis}
  \end{tikzpicture}
    \caption{\titleplot}
  \end{center}
\end{figure}



\section{Conclusion}
\blindtext


\end{document}
